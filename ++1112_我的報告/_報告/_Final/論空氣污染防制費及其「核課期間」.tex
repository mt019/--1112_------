\documentclass[11pt,a4paper]{article}

% \documentclass[UTF8,a4paper,14pt]{ctexart}
\usepackage[a4paper, margin={1in,1.5in}]{geometry}
\usepackage[fontsize=12pt]{fontsize}
\renewcommand{\footnotesize}{\fontsize{8pt}{11pt}\selectfont}
\usepackage{
  url}


\usepackage{xeCJK}
\usepackage{zhnumber}

\usepackage{titlesec}
\usepackage{titling}
\usepackage{fontspec}
\usepackage{newunicodechar}
\usepackage{tocloft} % adding the tocloft package for toc customization
\usepackage{enumitem}

\setcounter{tocdepth}{2} % table of content

\setcounter{secnumdepth}{5}

% note: I'm using different fonts only because I don't have yours
% \setCJKmainfont{Noto Serif CJK TC}
% \setCJKsansfont{Noto Sans CJK TC}
% \setCJKmonofont{Noto Mono CJK TC}


%中英文設定
%\usepackage{fontspec}
\setmainfont{TeX Gyre Termes}
\usepackage{xeCJK} %引用中文字的指令集
%\setCJKmainfont{PMingLiU}
\setCJKmainfont{DFKai-SB}
\setCJKmainfont[AutoFakeBold=4,AutoFakeSlant=.4]{DFKai-SB}   %設定軟體粗體及斜體
% \setmainfont{Times New Roman}
\setCJKmonofont{DFKai-SB}


\setlength{\parindent}{2em} %首行縮排兩個漢字距離
\usepackage{indentfirst}
% 預設第一段不首行縮排,如果想讓第一段首行縮排,則可以使用 \usepackage{indentfirst}。
% 如果想讓某一段不首行縮排,則可以在該段前加上 \noindent。
% 如果想讓整篇文章都首行不縮排,則:\setlength{\parindent}{0pt}


% in your example the titles in the toc are all sans serif, so I'll just add that here
% feel free to leave that out in your original document,
% it's just for visual comparability
\renewcommand{\cftsecfont}{\bfseries\sffamily}
\renewcommand{\cftsubsecfont}{\sffamily}
\renewcommand{\cftsubsubsecfont}{\sffamily}
\renewcommand{\cftparafont}{\sffamily}
\renewcommand{\cftsubparafont}{\sffamily}

% zhnum[style={Traditional,Financial}] doesn't work with the section counter,
% so we define our own counter and increase it every time in \thesection
\newcounter{mysec}[section]
\renewcommand\thesection{%
    \addtocounter{mysec}{1}%
    \zhnum[style={Traditional,Financial}]{mysec}、}
\renewcommand\thesubsection{\zhnum{subsection}、} % added a 、
\renewcommand\thesubsubsection{(\zhnum{subsubsection})} % added parentheses
% (full-width, don't know if that's what you want)
\renewcommand\theparagraph{} % you don't want paragraph numbers
\renewcommand\thesubparagraph{} % nor subparagraph numbers

% we have to adjust the spacing in the toc because the section label is longer than usual
\addtolength\cftsecnumwidth{1em}
\addtolength\cftsubsecindent{1em}
\addtolength\cftsubsubsecindent{1em}

% here we need to make sure the normal section counter is accessed
\titleformat{\section}{\Large\bfseries\filcenter}
    {\zhnum[style={Traditional,Financial}]{section}、}{.5em}{}
% not really sure what you intend to achieve with \fontsize but I'll leave it here
\titleformat*{\subsection}{\fontsize{15}{20}\bfseries\sffamily} 
\titleformat*{\subsubsection}{\fontsize{14}{18}\bfseries\sffamily}

% no extra version for numberless is necessary since no numbers are used anyways
% also you get newlines from omitting the [display] in \titleformat already
% \titleformat{\paragraph}
%     {\fontsize{14}{16}\bfseries\sffamily}{}{0em}{} 
% \titleformat{\subparagraph}
%     {\fontsize{12}{14}\bfseries\sffamily}{}{0em}{}
% we need the following so that they don't indent (second argument, 0em);
% you'll have to adjust the spacing though since this is not display style anymore:
% \titlespacing*{\paragraph}{0em}{3.25ex plus 1ex minus .2ex}{.75ex plus .1ex} 
% \titlespacing*{\subparagraph}{0em}{3.25ex plus 1ex minus .2ex}{.75ex plus .1ex}

% \renewcommand{\maketitlehooka}{\sffamily}

\renewcommand{\baselinestretch}{1.2}
\renewcommand{\abstractname}{摘要} 
\renewcommand{\contentsname}{\hfill\bfseries 目錄 \hfill} 
\renewcommand{\tablename}{表}
\renewcommand{\arraystretch}{1}


\usepackage{fancyhdr}%导入fancyhdr包

\usepackage{lastpage}
\pagestyle{fancy}
\fancyhf{} 
\cfoot{第 \thepage 頁,共\pageref*{LastPage} 頁}
\usepackage[hang,flushmargin,bottom]{footmisc} %
% \usepackage[]{footmisc}

% \usepackage{hyperref}
% \usepackage[utf8x]{inputenc} do not use inputenc with XeTeX
% \usepackage{fixltx2e} not required any more
\usepackage{graphicx}
\usepackage{longtable}
\usepackage{float}
\usepackage{wrapfig}
\usepackage{rotating}
\usepackage[normalem]{ulem}
\usepackage{amsmath}
\usepackage{textcomp}

\usepackage{multirow}
\usepackage{booktabs}

\usepackage{url}
\let\oldquote\quote
\let\endoldquote\endquote
\renewenvironment{quote}[2][]
  {\if\relax\detokenize{#1}\relax
     \def\quoteauthor{#2}%
   \else
     \def\quoteauthor{#2~---~#1}%
   \fi
   \oldquote}
  {\par\nobreak\smallskip\hfill(\quoteauthor)%
   \endoldquote\addvspace{\bigskipamount}}

   \usepackage{hyperref}
\hypersetup{
  colorlinks=true,
  linkcolor=[rgb]{0,0.37,0.53},
  citecolor=[rgb]{0,0.47,0.68},
  filecolor=[rgb]{0,0.37,0.53},
  urlcolor=[rgb]{0,0.37,0.53},
  % pagebackref=true, % this is ignored
  linktoc=all}

\usepackage{longtable}
\usepackage{array}
\usepackage{tabularray}
\newcolumntype{C}[1]{>{\centering\arraybackslash}p{#1}}


\usepackage{afterpage}
% \usepackage{ctex}
% \author{王逸帆}
% \date{\today}





\author{王逸帆\,
% R10A21126
\thanks{國立臺灣大學法律學系研究所碩士班財稅法學組二年級,學號:R10A21126。}
\vspace{-60em}
}
\date{}
% \date{\ctexset{today=big}}
\title{論空氣污染防制費及其「核課期間」
% \\  \large —— 以空氣污染防制費收費辦法第17條為中心 
\thanks{
  111學年度第2學期「稅法專題研究」課堂報告,授課教師:柯格鐘教授。}}


\setlength{\parskip}{1em}

\begin{document}

\maketitle
\makeatother

\vspace{1pt}

\begin{abstract}
\setlength{\parindent}{2em}
\noindent
\hspace*{0.9\parindent}
在現行財政收入之體系中,特別公課是非稅公課之一種類型。其中,具有環境管制及保護之目的者,屬於環境特別公課。作爲不同的公課類型,非稅特別公課與稅捐相比,二者之稽徵所依據之現行法規範體系不同,其財源收入之預算監督機制也不同。是否應以特別公課而非稅捐之概念定性相關之環境公課,不無爭議。而對於目前實務上所課徵之環境公課,相關法律規範仍需進一步完善。

本文關注環境公課,以「空氣污染防制費」爲例,聚焦「固定污染源空氣污染防制費」,梳理空氣污染防制法及收費辦法,檢視「追補繳」空污費之核課期間之實務做法及規範合理性,最後嘗試提出一些對於完善相關法律規範之期許。

   \end{abstract}



\thispagestyle{empty} %封面頁不編頁碼
\clearpage
    

\tableofcontents 


\thispagestyle{empty} %封面頁不編頁碼
\clearpage
\setcounter{page}{1} %從正文開始編頁碼


\section{前言}

% \subsection{環境公課}
% \subsection{空氣污染防制費之法律依據}
為防制空氣污染,維護生活環境及國民健康,以提高生活品質,立法院於1975年公布施行「空氣污染防制法」(下稱本法),歷經多次修法,對各空氣污染源徵收「空氣污染防制費」。行政院環保署依法律授權,訂定有「	空氣污染防制費收費辦法」(下稱收費辦法),亦經多次修正。
司法院大法官在釋字第 426 號解釋將「空氣污染防制費」定位為「特別公課」
\footnote{節錄該號解釋理由書:「憲法增修條文第九條(按:現移列第10條)第二項規定:「經濟及科學技術發展,應與環境及生態保護兼籌並顧」,係課國家以維護生活環境及自然生態之義務,防制空氣污染為上述義務中重要項目之一。空氣污染防制法之制定符合上開憲法意旨。依該法徵收之空氣污染防制費係本於污染者付費之原則,對具有造成空氣污染共同特性之污染源,徵收一定之費用,俾經由此種付費制度,達成行為制約之功能,減少空氣中污染之程度;並以徵收所得之金錢,在環保主管機關之下成立空氣污染防制基金,專供改善空氣品質、維護國民健康之用途。此項防制費既係國家為一定政策目標之需要,對於有特定關係之國民所課徵之公法上負擔,並限定其課徵所得之用途,在學理上稱為特別公課,乃現代工業先進國家常用之工具。
特別公課與稅捐不同,稅捐係以支應國家普通或特別施政支出為目的,以一般國民為對象,課稅構成要件須由法律明確規定,凡合乎要件者,一律由稅捐稽徵機關徵收,並以之歸入公庫,其支出則按通常預算程序辦理;特別公課之性質雖與稅捐有異,惟特別公課既係對義務人課予繳納金錢之負擔,故其徵收目的、對象、用途應由法律予以規定,其由法律授權命令訂定者,如授權符合具體明確之標準,亦為憲法之所許。」}。
解釋認為「空氣污染防制費」是一種特別公課,並
肯認特別公課係對於課徵義務人之公法上金錢負擔,作為一財政工具之類型,與稅捐公課為不同之財政工具,并且强調空氣污染防制費係本於污染者付費原則,有行為制約功能。


暫時不論學者對於空污費性質之不同見解
及對於廣徵特別公課現象爲害財政體系之擔憂
\footnote{參柯格鐘,特別公課之概念及爭議-以釋字第四二六號解釋所討論之空氣污染防制費為例,月旦法學雜誌,第 163 期 ,2008年11月,頁194-215。},
縱使經過多次修法,已消除了部分爭議,空氣污染防制費相關之法律規範體系依舊存在一些問題。
% 本文所選取之案例是關於固定污染源空氣污染防制費,故下文聚焦於此。
本文聚焦於固定污染源空氣污染防制費之短漏費,分析空氣污染防制法及收費辦法「追補繳」空污費之核課期間之實務做法及規範合理性。


\section{空氣污染防制費}

\subsection{污染者付費原則(略)}
(本節略,同上學期報告)
% 空污費之課徵涉及對於人民財產權等基本權利之限制,
% 須受法律保留原則之拘束,且應符合憲法第7條平等原則及第23條比例原則方屬合憲。
% 釋字第 426 號解釋確認空污費之課徵是本於污染者付費原則。文本試著將「污染者付費原則」理解爲憲法平等原則在環境公課之具體實現,對標作爲稅法基本原理原則之「量能課稅原則」。依據污染者付費原則,空污費之計費所應考量者應該是污染者之污染做造成的影響。
% 鑒於要以污染程度計算出污染者應該負擔之消除污染、復育環境和損害賠償之具體費用是困難複雜的問題,目前所徵收之空污費在具體數額上與各類污染之防治及環境復育費用
% % \footnote{環境基本法第28條:環境資源為全體國民世代所有,中央政府應建立環境污染及破壞者付費制度,對污染及破壞者徵收污染防治及環境復育費用,以維護環境之永續利用。}
% 難以相等同。空污費之課徵僅能在一定程度上作爲經濟誘因的管制手段而促使業者減少排放量
% \footnote{戴奧辛空污費遭批太低,環署:減排是目的,
% 見:https://www.cna.com.tw/news/ahel/201807290090.aspx.}。

% 由此,「污染者付費原則」對於空污費之課徵,雖然難以體現在絕對的費用計算(因所徵收之費額不一定相當於污染防治及環境復育等費用),但仍然應該符合平等原則。具體而言,對於不同之污染者,其空污費之計算應該依照法律規定,以污染源、空氣污染之類型、排放量等依據而計算。如此計算得到之公課負擔義務才是在環境公課的概念中,依據事物之本質而對污染者進行合理的差別對待,符合憲政法治國家平等原則之要求。


\subsection{徵收對象}

依據本法第16條第1項,各級主管機關得對排放空氣污染物之固定污染源及移動污染源徵收空氣污染防制費。依據第1款,對於固定污染源,徵收對象為污染源之所有人,其所有人非使用人或管理人者,向實際使用人或管理人徵收;其為營建工程者,向營建業主徵收;經中央主管機關指定公告之物質,得依該物質之銷售數量,向銷售者或進口者徵收。

\subsection{固定污染源空氣污染防制費計算方式(略)}
(本節略,同上學期報告)

% 空氣污染防制費收費辦法(以下簡稱收費辦法)第3條第4項規定,公私場所固定污染源排放之個別空氣污染物種類排放量任一季超過一公噸者,應即依第一項規定申報、繳費。另依收費辦法第4條,公私場所依第3條規定申報空氣污染防制費且其固定污染源排放二種以上空氣污染物者,應按其個別排放量計算費額,計算公式為:
% \begin{equation*}
%    \begin{aligned}
%      \text{空氣污染防制費費額}&=\sum \text{個別空氣污染物費額}\\
%      &=\sum\text{個別空氣污染物排放量}\times \text{收費費率。}
%    \end{aligned}
%  \end{equation*}


% 其中,空氣污染物排放量之計算依據,由空氣污染防制費收費辦法第10條規範。第10條第1項各款所條列之各計算依據順序如下:

% \begin{enumerate}[itemsep=0em]
%    \item 符合中央主管機關規定之固定污染源空氣污染物連續自動監測設施之監測資料。
%    % \item 符合中央主管機關規定之空氣污染物檢測方法之檢測結果。
%    \item 經中央主管機關認可之揮發性有機物自廠係數。
%    \item 符合中央主管機關規定之空氣污染物檢測方法之檢測結果,或中央主管機關指定公告之空氣污染物排放係數、控制效率、質量平衡計量方式。
%    \item 其他經中央主管機關認可之排放係數或替代計算方式。
% \end{enumerate}

% 依據污染者付費,空污費原則上應以實際之污染物排放量作爲計費依據。然而許多污染物的排放量沒辦法用儀器直接測量,即難以精確得知排放量,僅可依據原物料之使用量等因素通過推估計算得到。因此第10條第1項除了監測資料之外還有序列舉了其他的排放量計量依據。第10條第2項即對於公私場所申報固定污染源\textbf{揮發性有機物}排放量者有特別規範,
% 應以上述自廠係數、中央主管機關指定公告之空氣污染物排放係數、控制效率、質量平衡計量方式或其他經中央主管機關認可之排放係數或替代計算方式計算排放量。

% 以上所稱之收費費率,為空氣污染防制法第17條第2、3項規定授權予主管機關訂定並公告者\footnote{環署空字第1070050299號公告,依公私場所固定污染源排放空氣污染物之種類及排放量徵收空氣污染
% 防制費之收費費率,見:https://oaout.epa.gov.tw/law/LawContent.aspx?id=GL005189.},考慮固定污染源所處之防制區級別、排放之污染物種類、排放量之級別、季別等因素,且訂有優惠係數、減量係數,作爲計費之依據。需注意空污費費額之實際計費方式并非如收費辦法第 4 條以「 $\times$ 」符號所簡寫之關係,而是需要依據前述公告之收費費率及計算方式具體計算。為行文之便利,本文仍使用簡寫公式如上,先予説明。


\section{「追補繳」空污費之類型}


% 理論上來説,
所謂空污費之「追補繳」,發生在義務人所申報繳納之費額,與經過主管機關查核而得其應繳納之費額相比不足,或義務人未依規定進行申報之時。
% 具體而言,承接
短、漏報空污費之情形,經主管機關審查核算後之「追補繳」,依據義務人之主觀可歸責性主要區分爲兩種類型,適用不同條文,分別爲現行收費辦法第17條與現行本法第75條(舊101.09.06收費辦法18至19條)。本文主要依據法條文義以及立法理由作此區分,而兩種類型也有所適用污染源類型的差異。

\subsection{收費辦法第17條}

收費辦法第17條與舊(101.09.06)收費辦法的18至19條,所適用之污染源類型為本法第16條第1項第1款,也就是固定污染源(含營建工程)。

收費辦法第17條最新修法(111.03.24)明文規定了對於短漏空污費著進行重新核定之追溯期間(見表\ref{table.art.17})。
修正説明中表示,參考行政程序法第131條第1項規定之行政機關五年請求權,明定主管機關得以第9條第1項規定通知公私場所提報計算固定污染源空氣污染物排放量有關資料時間點之前一次申報季別,作為重新核定起始時間,往前追溯核定5年內應繳之空氣污染防制費。
% \begin{table}[b]%[htbp]
%     \centering
%     \caption{空氣污染防制費收費辦法第17條修正條文對照}
%     \label{table.1}
%     \begin{tabular}{p{0.3\linewidth}|p{0.3\linewidth} | p{0.45\linewidth}}
%         % {lll} 
%     \toprule
%     修正條文(111.3.24)                                                                                                                                                                                                                                                                                                                                                                                                                                                                                                 & 修正前條文(101.09.06)                                                                                                                                                                                                                                                                                                                                                                                                                                              & 說明                                                                                                                                                                                                                                                                                                                                 \\ 
%     \hline
%     \begin{tabular}[c]{p{\linewidth}}公私場所依第三條規定應申報空氣污染防制費,有下列情形之一,中央主管機關得逕依其固定污染源產品產量、原(物)料使用量、燃料使用量、檢測結果、連續自動監測設施原始數據或其他有關資料,自依第九條第一項規定通知公私場所提報時間之前一次申報季別起,計算追溯五年內其固定污染源空氣污染物排放量,重新核定其應繳之空氣污染防制費:\\一、未依規定計算空氣污染物排放量之情形。\\二、因設施故障或其他因素,致無法維持正常操作或廢氣未經收集或防制設施處理即排放於大氣中,未計算空氣污染物排放量。\\三、未於第九條規定期限內提報空氣污染物排放量相關資料、其補正資料不足。\\四、產品產量、原(物)料、燃料使用量與其購買量及結算結果不符。\\五、申報之固定污染源數量與實際情形不符。\\六、經中央主管機關查核有第十一條或十二條之情形。\\七、其他未依規定申報空氣污染防制費。\\營建業主未依第五條、第七條規定申報、調整空氣污染防制費或提供資料不完整者,直轄市、縣(市)主管機關得逕依查驗結果或相關資料,核定其應繳納之空氣污染防制費。\end{tabular} & \begin{tabular}{p{\linewidth}}公私場所依第三條規定應申報空氣污染防制費,有下列情形之一,中央主管機關得逕依其固定污染源產品產量、原(物)料使用量、燃料使用量、檢測結果或其他有關資料,計算其固定污染源空氣污染物排放量,核定其應繳納之空氣污染防制費:\\一、未依規定計算空氣污染物排放量之情形。\\二、因設施故障或其他因素,致無法維持正常操作或廢氣未經收集或防制設施處理即排放於大氣中,未計算空氣污染物排放量。\\三、未於第九條規定期限內提報空氣污染物排放量相關資料、其補正資料不足。\\四、產品產量、原(物)料、燃料使用量與其購買量及結算結果不符。\\五、申報之固定污染源數量與實際情形不符。\\六、經中央主管機關查核有第十一條或十二條之情形。\\七、其他未依規定申報空氣污染防制費。\\營建業主未依第五條、第七條規定申報、調整空氣污染防制費或提供資料不完整者,直轄市、縣(市)主管機關得逕依查驗結果或相關資料,核定其應繳納之空氣污染防制費。\end{tabular} & \begin{tabular}{p{\linewidth}}一、第一項各款內容未修正,第一項序文修正說明如下:\\(一)配合第九條第一項第五款增訂連續自動監測設施查核所需資料之規定,修正重新核定應繳費額之連續自動監測設施監測數據之文字內容。\\(二)為明確重新核定之追溯期限,參考行政程序法第一百三\\十一條第一項規定之行政機關五年請求權作法,明定主管機關得以第九條第一項規定通知公私場所提報計算固定污染源空氣污染物排放量有關資料時間點之前一次申報季別,作為重新核定起始時間,往前追溯核定五年內應繳之空氣污染防制費;另倘主管機關因公私場所提報資料不全,須多次通知業者補件者,仍以第一次通知時間點為準。\\二、第二項未修正。\end{tabular}  \\
%     \bottomrule
%     \end{tabular}
%     \end{table}

\afterpage{
    \clearpage

    \begin{longtable}{C{\dimexpr 0.3\linewidth-2\tabcolsep}|C{\dimexpr 0.22\linewidth-2\tabcolsep} | C{\dimexpr 0.5\linewidth-2\tabcolsep}}
        % \caption{空氣污染防制費收費辦法第17條修正條文對照\label{table.art.17}}\\ \endlastfoot
        \toprule
        % 修正條文(111.3.24) & 修正前條文(101.09.06)
        \begin{tabular}[c]{@{}c@{}}\textbf{修正條文}\\\textbf{(111.03.24)}\end{tabular}                                                                                                                                                                       & \begin{tabular}[c]{@{}c@{}}\textbf{修正前條文}\\\textbf{(101.09.06)}\end{tabular} 
          & \textbf{說明}  \endfirsthead 
        \hline
        \begin{tabular}[c]{p{0.9\linewidth}}公私場所依第三條規定應申報空氣污染防制費,有下列情形之一,中央主管機關得逕依其固定污染源產品產量、原(物)料使用量、燃料使用量、檢測結果、連續自動監測設施原始數據或其他有關資料,\textbf{自依第九條第一項規定通知公私場所提報時間之前一次申報季別起},計算\textbf{追溯五年內}其固定污染源空氣污染物排放量,\textbf{重新}核定其應繳之空氣污染防制費:\\(略)\\\end{tabular} & \begin{tabular}[c]{p{0.9\linewidth}}公私場所依第三條規定應申報空氣污染防制費,有下列情形之一,中央主管機關得逕依其固定污染源產品產量、原(物)料使用量、燃料使用量、檢測結果或其他有關資料,計算其固定污染源空氣污染物排放量,核定其應繳納之空氣污染防制費:\\(略)\\\end{tabular} & \begin{tabular}[c]{p{\linewidth}}一、第一項各款內容未修正,第一項序文修正說明如下:\\(一)配合第九條第一項第五款增訂連續自動監測設施查核所需資料之規定,修正重新核定應繳費額之連續自動監測設施監測數據之文字內容。\\(二)為明確重新核定之追溯期限,參考行政程序法第一百三十一條第一項規定之行政機關五年請求權作法,明定主管機關得以第九條第一項規定通知公私場所提報計算固定污染源空氣污染物排放量有關資料時間點之前一次申報季別,作為重新核定起始時間,往前追溯核定五年內應繳之空氣污染防制費;另倘主管機關因公私場所提報資料不全,須多次通知業者補件者,仍以第一次通知時間點為準。\\\addlinespace 二、第二項未修正。\end{tabular}  \\
        \bottomrule
        \addlinespace
        \caption{空氣污染防制費收費辦法第17條修正條文對照\label{table.art.17}}\\ 
        \end{longtable}
  }      



        % \usepackage{tabularray}

        % \usepackage{tabularray}
% \begin{longtblr}[
%     caption = {空氣污染防制費收費辦法第17條修正條文對照},
%     label = {table.1},
%   ]{
%     row{2} = {t},
%     hlines,
%     hline{1,3} = {-}{0.08em},
%   }
%   修正條文(111.3.24)                                                                                                                                                       & 修正前條文(101.09.06)                                                                                                    & 說明                                                                                                                                                                                                                                                                                          \\
%   {公私場所依第三條規定應申報空氣污染防制費,有下列情形之一,中央主管機關得逕依其固定污染源產品產量、原(物)料使用量、燃料使用量、檢測結果、連續自動監測設施原始數據或其他有關資料,自依第九條第一項規定通知公私場所提報時間之前一次申報季別起,計算追溯五年內其固定污染源空氣污染物排放量,重新核定其應繳之空氣污染防制費:\\(略)} & {公私場所依第三條規定應申報空氣污染防制費,有下列情形之一,中央主管機關得逕依其固定污染源產品產量、原(物)料使用量、燃料使用量、檢測結果或其他有關資料,計算其固定污染源空氣污染物排放量,核定其應繳納之空氣污染防制費:\\(略)} & {一、第一項各款內容未修正,第一項序文修正說明如下:\\(一)配合第九條第一項第五款增訂連續自動監測設施查核所需資料之規定,修正重新核定應繳費額之連續自動監測設施監測數據之文字內容。\\(二)為明確重新核定之追溯期限,參考行政程序法第一百三\\十一條第一項規定之行政機關五年請求權作法,明定主管機關得以第九條第一項規定通知公私場所提報計算固定污染源空氣污染物排放量有關資料時間點之前一次申報季別,作為重新核定起始時間,往前追溯核定五年內應繳之空氣污染防制費;另倘主管機關因公私場所提報資料不全,須多次通知業者補件者,仍以第一次通知時間點為準。\\二、第二項未修正。} 
%   \end{longtblr}

\subsection{以故意方式短、漏報者}

本法第75條之立法,係提升舊空氣污染防制費收費辦法(101.09.06)第18、19條之法律規範位階而來(條文對照如表\ref{article-compare})。
現行本法第75條所適用之污染源類型為本法第16條第1項,也就是包括了固定污染源及移動污染源。
適用本法第75條第1項規定之義務人為「有偽造、變造或其他不正當方式短報或漏報與空氣污染防制費計算有關資料者」。111年3月24日刪除前收費辦法第18條之修法理由為:「一、條次變更。二、本條文係規範公私場所蓄意以不實資料申報或虛偽記載、變造與空氣污染防制費相關資料者,主管機關得逕依2倍排放量或費率計算空氣污染防制費,爰於修正條文增列『以故意方式短報或漏報』之相關文字,『以排除非故意短報或漏報者』。……」可知該條文明定短報或漏報行為須故意者。


空氣污染防制法第75條第1項對於義務人以故意方式短報或漏報者,規定應由各主管機關以倍數計算其「應繳費額」\footnote{本文暫不討論其詳細計算方式及法律爭議}。第75條第2項規定了5年的追溯期間,然并未如現行收費辦法一樣明確規定期間之起算時點。




\begin{table}[b]%[htbp]
    \centering
    \begin{tabular}{p{0.5\linewidth} | p{0.45\linewidth}}
    \hline
    \textbf{空氣污染防制法 (107.08.01)}  
     & \textbf{空氣污染防制費收費辦法 (101.09.06)}   \\ 
    \hline
    \multirow[t]{2}{0.9\linewidth}[-11em]{\begin{tabular}[c]{p{\linewidth}}\textbf{第 75 條}\\
        公私場所依第十六條第一項繳納空氣污染防制費,有偽造、變造或其他不正當方式短報或漏報與空氣污染防制費計算有關資料者,各級主管機關應依下列規定辦理:\\\begin{tabular}[c]{p{0.9\linewidth}}一、移動污染源:中央主管機關得逕依移動污染源空氣污染防制費收費費率之二倍計算其應繳費額。\\二、營建工程:直轄市、縣(市)主管機關得逕依查驗結果或相關資料,以營建工程空氣污染防制費收費費率之二倍計算其應繳費額。\\三、營建工程以外固定污染源:中央主管機關得逕依排放係數或質量平衡核算該污染源排放量之二倍計算其應繳費額。\\[5pt]\end{tabular}\\公私場所以前項之方式逃漏空氣污染防制費者,各級主管機關除依前\textbf{條}計算及徵收逃漏之空氣污染防制費外,並追溯五年內之應繳費額。但應徵收空氣污染防制費之空氣污染物起徵未滿五年者,自起徵日起計算追溯應繳費額。\\[5pt]前項追溯應繳費額,應自各級主管機關通知限期繳納截止日之次日或逃漏空氣污染防制費發生日起,至繳納之日止,依繳納當日郵政儲金一年期定期存款固定利率按日加計利息。\end{tabular}} & 
        
        
        \begin{tabular}[c]{p{0.9\linewidth}}\\\textbf{第 18 條}\\公私場所依本法第十六條第一項第一款繳納空氣污染防制費之固定污染源,有偽造、變造或以故意方式短報或漏報與空氣污染防制費計算有關資料者,各級主管機關得依下列規定辦理:\\\begin{tabular}[c]{{p{0.9\linewidth}}}一、中央主管機關得逕依排放係數核算該污染源排放量之二倍計算空氣污染防制費。\\二、其為營建工程者,直轄市、縣(市)主管機關得逕依查驗結果或相關資料以營建工程空氣污染防制費收費費率之二倍計算其應繳費額。\end{tabular}\end{tabular} \\ \\
    \cline{2-2}                                                                                              & \begin{tabular}[c]{p{0.9\linewidth}}\\\textbf{第 19 條}\\公私場所以前條之方式或其他不正當方法逃漏空氣污染防制費者,中央主管機關得重新計算追溯五年內之應繳金額。應徵收空氣污染防制費之空氣污染物起徵未滿五年者,則自起徵日起計算追溯應繳金額。\\[5pt]前項追溯應繳金額,應自主管機關通知限期繳納截止日之次日或逃漏空氣污染防制費發生日起,至繳納之日止,依繳納當日郵政儲金一年期定期存款固定利率按日加計利息。\end{tabular}            
    \\\\
    \hline
    \end{tabular}
    \caption{\label{article-compare}空氣污染防制法及空氣污染防制費收費辦法相關條文對照}
    \end{table}




    




\section{關於「核課期間」之實務案例}

% \makeatletter
% \newcommand\urlfootnote@[1]{\footnote{\url@{#1}}}
% \DeclareRobustCommand{\urlfootnote}{\hyper@normalise\urlfootnote@}
% \makeatother

爲了具體化「追補繳」空污費相關規範之問題,
本文針對關於「追補繳」空污費之案件:
% 臺塑關係企業案\footnote{\href{https://judgment.judicial.gov.tw/FJUD/data.aspx?ty=JD\&id=TPAA\%2c103\%2c\%e5\%88\%a4\%2c216\%2c20140430\%2c1}{最高行政法院 103 年度判字第 216 號判決}}
% 台塑二案\footnote{\href{https://judgment.judicial.gov.tw/FJUD/data.aspx?ty=JD&id=TPAA\%2c109\%2c\%e5\%88\%a4\%2c55\%2c20200206\%2c1&ot=in}{最高行政法院 109 年度判字第 55 號判決}} 、
信鼎案\footnote{\href{https://judgment.judicial.gov.tw/FJUD/data.aspx?ty=JD&id=TPBA\%2c107\%2c\%e8\%a8\%b4\%e6\%9b\%b4\%e4\%b8\%80\%2c14\%2c20191114\%2c1}{臺北高等行政法院 107 年度訴更一字第 14 號判決}} 、 
台塑案\footnote{\href{https://judgment.judicial.gov.tw/FJUD/data.aspx?ty=JD&id=TPAA\%2c109\%2c\%e5\%88\%a4\%2c54\%2c20200206\%2c1&ot=in}{最高行政法院 109 年度判字第 54 號判決}} 、
宏全案\footnote{\href{https://judgment.judicial.gov.tw/FJUD/data.aspx?ty=JD&id=TPAA\%2c109\%2c\%e4\%b8\%8a\%2c1125\%2c20221117\%2c1}{最高行政法院 109 年度上字第 1125 號判決}} 、沛鑫案\footnote{\href{https://judgment.judicial.gov.tw/FJUD/data.aspx?ty=JD&id=TPAA\%2c111\%2c\%e4\%b8\%8a\%2c393\%2c20220616\%2c1}{最高行政法院 111 年度上字第 393 號裁定}}、
% 三櫻案\footnote{\href{https://judgment.judicial.gov.tw/FJUD/data.aspx?ty=JD&id=TPAA\%2c110\%2c\%e4\%b8\%8a\%2c121\%2c20220530\%2c1}{最高行政法院 110 年度上字第 121 號判決}}
三櫻案\footnote{\href{https://judgment.judicial.gov.tw/FJUD/data.aspx?ty=JD&id=TCBA\%2c110\%2c\%e8\%a8\%b4\%2c21\%2c20220830\%2c1&ot=in}{臺中高等行政法院 110 年度訴字第 21 號判決}}等進行整理(見表\ref{案例整理}),發現實務上一重要爭議在於「追補繳」之請求權時效。本文以最新的「三櫻案」\footnote{收費辦法第17條修法(111.03.24)之前。}爲例。

\afterpage{
% \usepackage{tabularray}
\begin{table}[h]
     \centering
     % \caption{案例}
     \begin{tblr}{
       cell{3}{1} = {r=4}{},
       cell{3}{2} = {r=4}{},
       cell{8}{1} = {r=2}{},
       cell{8}{2} = {r=2}{},
       cell{10}{1} = {r=2}{},
       cell{10}{2} = {r=2}{},
       vlines,
       hline{1,12} = {-}{0.08em},
       hline{2-3,7-8,10} = {-}{},
       hline{4-6,9,11} = {3-5}{},
     }
          & \textbf{查核時點} & \textbf{追補處分} & \textbf{處分作成時點} & \textbf{追補費期間}  \\
     信鼎案  & 100年4月24日     & 原處分            & 103年10月28日~    & 095年Q2-100年Q2   \\
     台塑案 & 101年1月、3月     & 原處分一           & 105年5月7日~      & 099年Q4(麥寮三廠)    \\
          &               & 原處分二           & 105年5月20日~     & 099年Q3-Q4(麥寮一廠) \\
          &               & 原處分三           & 105年6月6日~      & 100年Q1-Q4(麥寮一廠) \\
          &               & 原處分四           & 105年6月6日~      & 100年Q1-Q4(麥寮三廠) \\
     宏全案  & 107年9月  & 原處分            & 108年5月2日        & 102年Q3-107年Q2   \\
     沛鑫案  & 107年9月19日     & 前處分            & 108年5月2日        & 102年Q3-107年Q2   \\
          &               & 原處分            & 110年3月2日        & 103年Q2-107年Q2   \\
     三櫻案  & 107年9月26日     & 前處分            & 108年9月3日        & 102年Q3-107年Q2   \\
          &               & 原處分            & 109年6月22日       & 102年Q3-107年Q2   
     \end{tblr}
     \caption{案例整理}
     \label{案例整理}
     \end{table}}

\subsection{案件背景}

原告三櫻公司為固定污染源(非營建工程)之所有人,為本法第16條第1項第1款之義務人,申報繳納空污費。被告臺中市政府環保局於民國107年9月26日進廠查核,發現原告對於各類原物料有短報或漏報空氣污染防制費情事。被告機關乃依空氣污染防制法第75條、空氣污染防制費收費辦法第14條第1項等規定,計算並追溯5年內(即自102年第3季至107年第2季)之應繳費額,以108年9月3日中市環空字第1080099382號函(下稱前處分)命原告應補繳新臺幣(下同)227,903,501元。原告不服,提起訴願,前經臺中市政府以109年3月2日府授法訴字第1080247201號作成「原處分撤銷,由被告機關於收受決定書之次日起30日內另為適法之處分」訴願決定(下稱前訴願決定)在案。嗣被告機關重新查核後,以109年6月22日中市環空字第1090067952號函命(下稱原處分)原告應補繳102年第3季至107年第2季止之固定污染源空氣污染防制費共計163,282,519元,原告不服,提起訴願,經訴願決定駁回,原告仍不服,遂向臺中高等行政法院提起本件行政訴訟。

\subsection{時效爭議}
本案重要爭點之一即關於空污費之公法上請求權時效。具體而言,時效之起算時點、中斷時點等因素極大程度地影響了義務人在實體上的權利義務。
\subsubsection{原告主張}
本案原告對於原處分關於時效的主張:原處分追繳原告104年6月22日前之空污費,其請求權已罹於5年消滅時效,應屬無據。


收費辦法第17條修法之前,關於時效之規定應適用行政程序法第131條以下之規定。第131條第1項第1分句規定,公法上之請求權,於請求權人為行政機關時,除法律另有規定外,因五年間不行使而消滅。第131條第2項規定時效因行政機關為實現該權利所作成之行政處分而中斷。 第132條規定,行政處分因撤銷、廢止或其他事由而溯及既往失效時,自該處分失效時起,已中斷之時效視為不中斷。

被告對於原告102年第3、4季、第103年第1、2、3季各季空污費之請求權分別自102年10月1日、103年1月1日、4月1日、7月1日、10月1日起算,末日分別為107年9月30日、107年12月31日、108年3月31日、6月30日、9月30日,

被告雖曾作成前處分,命原告補繳空污費227,903,501元,惟該處分既經訴願決定撤銷,依行政程序法第132條規定,時效視為不中斷。依此,原處分僅能就109年6月22日前之5年內空污費重新核算並命原告補繳,即104年6月22日前之請求權已罹於5年消滅時效。

\subsubsection{被告抗辯}

被告抗辯為原處分未罹於時效。被告主張,本件空污費之公法上請求權,在原處分核算作成前尚未發生,公法上消滅時效無從起算,故原處分並無罹於時效問題,原告主張顯無理由。

被告之抗辯理由,本質上是關於空污費核課處分之性質及法律效果。被告認爲,核課處分之作成才使得空污費之請求權發生,也就是報告認爲該核課處分屬於形成處分,而非確認處分。


\subsubsection{法院見解}

臺中高等行政法院認爲,本件被告對原告所為固定污染源空氣污染防制費之補繳請求權除102年第3季外,未罹於消滅時效。其理由大致為:
依空氣污染防制法第75條第2項、101年9月6日修正公布之收費辦法第19條第1項之規定,主管機關得追溯「5年內」應繳空氣污染防制費之金額,並自起徵日起計算追溯應繳金額。次按「公法上之請求權,於請求權人為行政機關時,除法律另有規定外,因5年間不行使而消滅;於請求權人為人民時,除法律另有規定外,因10年間不行使而消滅。公法上請求權,因時效完成而當然消滅。前項時效,因行政機關為實現該權利所作成之行政處分而中斷。」行政程序法第131條定有明文;而空污費之徵收或追繳均為特別公課,屬公法上之請求權,應有上開5年請求權時效規定之適用,且上開修正前收費辦法第19條第1項規定以故意方式或其他不正當方法逃漏空氣污染防制費者,其追溯繳納空污費之期間為5年,性質上即與公法上請求權時效期間相同。又「消滅時效,自請求權可行使時起算。」為民法第128條所明定。所謂請求權「可行使」時,係指請求權人行使其請求權,客觀上無法律上之障礙而言,要與請求權人主觀上何時知悉其可行使無關(最高行政法院107年判字第37號判決意旨可參);應認被告所得追溯請求原告補繳未逾5年之空氣污染防制費,係自其客觀上無法律上之障礙而得行使請求權時起。查被告於107年9月26日對原告稽查時,已就巡查目的——空污費催補繳或查核,就檢測報告、會計資料、物料清冊、操作數量等進行查核作業,
% 並經原告代表即承辦員陳子源簽名在卷可稽,
應認被告業將其行使追溯繳納空污費之請求權通知原告在案,至於確切之補繳金額則待調查後再確認,故自該時即107年9月26日稽查起即屬已行使補繳空氣污染防制費請求權。

法院針對時效之見解可總結爲兩部分:第一,法院認爲時效之起算應為申報期間屆滿
之翌日,即每年4、7、10月及次年1月底前申報及繳納空氣污染防制費之時點開始計算時效;第二,主管機關之查核屬於時效中斷之事由。
% 被告於107年9月26日對原告稽查時起即屬已行使補繳空氣污染防制費請求權。
因此,依據行政程序法第131條第3項、第132條、第133條,考量時效中斷與重行起算,法院認爲被告命補繳102年第4季至107年第2季之空氣污染防制費之部分合法。

% 按空污費之徵收與追繳,性質上屬特別公課,依行政程序法第131條第1項前段規定,應適用5年時效規定。而依行為時收費辦法第3條第1項規定,繳納義務人應於每年4月、7月、10月及次年1月底前自行向指定金融機構繳納空污費,惟行為時空污法第20條第1項所稱空污費之「起徵日」,究指每年4月、7月、10月及次年1月之第1日或月底日,法未明指,自應類推適用稅捐稽徵法第22條規定,即已在規定期間內申報繳納者,自申報日起算;未在規定期間內申報繳納者,自規定申報期間屆滿之翌日起算。

% \subsubsection{本文見解}


\section{本文見解}


承接對於上述實務案例之分析,本文對於現行法制關於空氣污染防制費之核課期間之規範有以下見解:



\subsection{時效起算時點}

% 「消滅時效,自請求權可行使時起算。」為民法第 128 條所明定。所謂請求權「可行
% 使」時,係指請求權人行使其請求權,客觀上無法律上之障礙而言,要與請求權人主
% 觀上何時知悉其可行使無關(最高行政法院 107 年判字第 37 號判決意旨可參);應認
% 被告所得追溯請求原告補繳未逾 5 年之空氣污染防制費,係自其客觀上無法律上之障
% 礙而得行使請求權時起。


  

% 特別公課之公法上債權債務關係依法律規定而產生,時效起算時點
% 應自其客觀上無法律上之障礙而得行使請求權時起。
% 本文認爲應該

% 分析被告之抗辯理由,顯見被告臺中市政府環保局對於空污費之性質,并不認爲其屬於法定之債。依據被告之觀點,

本文大致認同上述法院關於請求權時效起算時點之見解,認爲時效之起算應自其客觀上無法律上之障礙而得行使請求權時起。具體而言,時效之起算應類似稅捐稽徵法第22條規定,已在規定期間內申報繳納者,自申報日起算;未在規定期間內申報繳納者,自規定申報期間屆滿之翌日起算。此時點即屬客觀上可合理期待主管機關得為追繳時。也就是説,核課期間應爲自每年4、7、10月及次年1月底前申報及繳納空氣污染防制費之時點或申報繳納期間屆滿之翌日開始計算。



\subsection{時效中斷時點}



本文不認同法院見解之部分爲,法院稱稽查時點屬已行使補繳空氣污染防制費請求權。本文不認爲主管機關之查核屬於時效中斷之事由。依據行政程序法第131條第3項規定,公法上請求權時效因行政機關為實現該權利所作成之行政處分而中斷。蓋主管機關依收費辦法第9條之規定行使職權進行實地查核,此非屬於主管機關為實現空污費之債權的行政處分的作成。雖然實地查核是主管機關關於空污費之徵收而進行職權之履行,即便查核作業之執行需要命義務人提報相關資料而構成行政處分,但查核之執行僅爲資料之搜集,雖有助於主管機關對於義務人應納空污費之計算,卻並不涉及具體數額之確定,難謂屬於行政程序法第131條第3項規定之時效中斷事由。總言之,本文認爲查核非請求權之行使。
請求權時效應依據行政程序法第131條,因行政機關為實現該權利所作成之行政處分(命補繳之處分,而非查核)而時效中斷。

\subsection{請求權時效的法律保留原則}


現行收費辦法第17條明定主管機關得以第 9 條第 1 項規定通知公私場所提報計算固定
污染源空氣污染物排放量有關資料時間點之前一次申報季別,作為重新核定起始時間,
往前追溯核定 5 年內應繳之空氣污染防制費。此次修法,本質上,與上述臺中高等行政法院之見解相同,即認爲主管機關查核之執行(命提報相關資料)屬於行政機關為實現該權利所作成之行政處分,構成行政程序法第131條第3項之時效中斷事由。而如此規範,實質上延長了核課期間,不利於義務人之實體法上的權利。
然而多次的局部修法導致本法與收費辦法的規範不一致。例如,現行
第 75 條第 2 項規定了 5 年的追溯期間,然并未如現行收費辦法一樣明確規定期間之起算時點。

本文認爲現行收費辦法第17條所定之時效不合法。
依大法官釋字723號解釋,時效制度須受絕對法律保留原則之限制。而在
在相對法律保留原則之下,立法者授予中央主管機關一定決定空間之事項,應僅限於高度專業性、技術性之事項。
核課期間,攸關人民之基本權利而非屬於高度專業性及技術性之事項,若以收費辦法之形式被規範,且不利於人民之基本權利,則與法律保留原則有違。
總言之,本文認爲,為符合法律保留原則,
空氣污染防制法對於空污費之徵收,僅得就涉及高度專業性及技術性之事項授權予中央主管機關以收費辦法之形式規範,方符合法律保留原則,遵循憲法保障人民基本權利之意旨。而收費辦法第17條,以查核作爲時效中斷之事由不合法。


\section{小結}

% \subsection{現行規範之不足}


% \subsubsection{應該明確空污費短漏之法律效果}

% 重新核算之計費方式

% 時效制度之具體規範

% \subsection{應將空污費改以指定用途稅之形式課徵}

承接以上對於實務案例與現行規範不足之處之分析,
本文認爲,長期來看,應將空污費改以指定用途稅之形式課徵。
費改稅的過程需要涉及對於現行財稅制度的全面檢討,有助於健全財政法治。考慮到空污費作爲環境公課,是環境管制的一環,其構成要件事實與法律之適用需要環境主管機關之專業知識。
然而,為遵循法律保留原則,立法者儘得將涉及高度專業性、技術性之事項授權予中央主管機關以行政命令之形式加以規範。

改以指定用途稅之形式課徵空污稅,即可適用稅捐稽徵法關於稅捐債權之消滅時效期間,包括核課期間與徵收期間。核課期間及期間起算之規定,並采時效不完成制度,不適用行政程序法第131第3項至134條有關時效中斷之規定。具體而言,空污稅為義務人自行申報繳納之稅捐,適用稅捐稽徵法第22條第1及第2款之核課期間之起算時點,且適用稅捐稽徵法第22條有關徵收期間之規定。

此外,碳費、碳稅制度正在研議之中。若能首先將空污費改以指定用途稅之形式課徵,則在一定程度上能幫助後續有效建立碳稅制度,有助整體財稅體系的完善。
% 全面檢討現行稅制

% 空污費以指定用途稅之形式課徵



% \subsection{本文假設特別公課為法定之債}
% \subsubsection{法定之債}

% 將空污費改以指定用途稅之形式課徵


% 應該明確空污費為法定之債,即是明確本文所討論的核課期間,為請求權時效期間。

% 特別公課與稅捐的區別,主要在於群體公益性、群集有責性。

% 理由:在管制手段的角度,采取課徵空污費或以課徵(指定用途)稅的選擇,應該只是立法上的選擇,不應該


% \subsubsection{遵循法律保留原則}



% \subsubsection{核定與徵收期間}



% \subsubsection{整合碳稅}




\end{document}